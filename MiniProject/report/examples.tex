%Give code examples where your extension is useful, and describe how they work
%with it. Make sure you include examples where the most intricate features of
%your extension are used, so that we have an immediate understanding of what the
%challenges are.
%
%You can pretty-print tool code like this:
%\begin{lstlisting}
%object {
%  def main() : Unit = { println(new A().foo(-41)); }
%}
%
%class A {
%  def foo(i : Int) : Int = {
%    var j : Int;
%    if(i < 0) { j = 0 - i; } else { j = i; }
%    return j + 1;
%  }
%}
%\end{lstlisting}
%
%This section should convince us that you understand how your extension can be
%useful and that you thought about the corner cases.

We now give code examples that show where our extension is useful.\\

\begin{lstlisting}[caption={This program prints true, false, true}]
object BoolArray {
    def main() : Unit = {
        println(new A().foo());
    }
}

class A {
    def foo(): Bool = {
        var ba: Bool[];
        ba = new Bool[3];
        ba[0] = true;
        ba[1] = false;
        ba[2] = ba[0];
        println(ba[0]);
        println(ba[1]);
        
        return ba[2];
    }
}
\end{lstlisting}

\begin{lstlisting}[caption={VarArgs: this program prints "bar"}]
object StringVarArg {
    def main() : Unit = {
        println(new A().foo("b", "a", "r"));
    }
}

class A {
    def foo(array: String*): String = {
		var i: Int = 0;
        var a: String = "";
        while (i < array.length) {
            a = a + array[i];
            i = i + 1;
        }
		return a;
    }
}
\end{lstlisting}

\begin{lstlisting}[caption={Default arguments: this program prints "m: 6 n: 7 o: bar, m: 1 n: 2 o: foo"}]
object DefaultArg {
    def main() : Unit = {
        println(new A().foo(6));
        println(new A().foo(1, 2, "foo"));
    }
}

class A {
    def foo(m: Int, n: Int = 7, o: String = "bar"): Bool = {
        println("m: " + m + " n: " + n + " o: " + o);
        return true;
    }
}
\end{lstlisting}


