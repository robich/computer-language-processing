We explain in this section how we implemented these extensions.

%\subsection{Theoretical Background}
%If you are using theoretical concepts, explain them first in this subsection.
%Even if they come from the course (eg. lattices), try to explain the essential
%points \emph{in your own words}. Cite any reference work you used like this
%\cite{TigerBook}. This should convince us that you know the theory behind what
%you coded. 
%
%\subsection{Implementation Details}
%Describe all non-obvious tricks you used. Tell us what you thought was hard and
%why. If it took you time to figure out the solution to a problem, it probably
%means it wasn't easy and you should definitely describe the solution in details
%here. If you used what you think is a cool algorithm for some problem, tell us.
%Do not however spend time describing trivial things (we what a tree traversal
%is, for instance). 
%
%After reading this section, we should be convinced that you knew what you were
%doing when you wrote your extension, and that you put some extra consideration
%for the harder parts.

\subsection{Bool, String and Object arrays}
This step was obviously needed to support varargs.\\
We began by enhancing the parser to be able to parse Bool[], String[], $<$Identifier$>$[] into the new trees that we added. In general this step was straightforward, but we had to read the \href{http://docs.oracle.com/javase/specs/jvms/se7/html/jvms-4.html\#jvms-4.9.1-120-O}{JVM doc} describing class file format to find what value was to be given to the \href{https://github.com/psuter/cafebabe}{cafebabe} $NewArray$ abstract byte code, depending on what type the array was supposed to contain.

\subsection{VarArgs}
The parser was modified to recognize VarArgs (denoted by a trailing star). We added a default boolean value to the $Formal$ $Tree$ that we use both in the name analysis and in the code generation phases.\\
We chose to allow only a single VarArg per method declaration. Moreover, if used, it must be defined as the last parameter of the method. This choice simplifies greatly the implementation and follows the decision of Java and Scala.\\
When adding a method to the cafebabe $ClassFile$, if the method has a VarArg, we specify this argument type to be the one of an array ($IntArray$, $BoolArray$ or $ObjectArray$).
On a $MethodCall$, if the method has a VarArg, we create an array containing the parameters linked to the VarArg.

\subsection{Default args}
For the sake of simplicity we forbid a formal to be both a vararg and have a default value.
We compare the difference between the number of arguments (formals) of a method declaration to the ones of a method call to know if a default value needs to be used. If such is the case, we insert the default value at the correct position in the method call.

\subsection{Default members}
Variable declarations can be assigned a default value using syntax of the form $var$ $name: Type = expression$. The default value is stored in the variable symbol. The code generation phase then keeps track of which variables have been initialized, and if an uninitialized variable is accessed its default value is used.