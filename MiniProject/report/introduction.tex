In the first part of this computer language processing project we built a simple compiler for TOOL (Toy Object Oriented Language). The pipeline through which an input TOOL program goes is composed of the following components: lexer (yields tokens from chars), parser (yields Abstact Syntax Trees, AST from tokens), name analyser (rejects bad inputs due to errors such as defining a class twice or overloading a method), type checker (rejects bad inputs due to type errors), code generator (yields jvm bytecode from ASTs). \\ \\
The extension we've chosen to work on will enable the use of \begin{itemize}
\item default scala-like method parameters
\item default values for class and method variables
\item arbitrary number of arguments to a function and therefore Bool, String and Object arrays.
%\item an extra copy() method that allows a quick cloning of an object while specifying which field to modify 
\end{itemize}

%This section should convince us that you have a clear picture of the general
%architecture of your compiler and that you understand how your extension fits
%in it.
